\documentclass[a4paper,landscape,10pt]{cheatsheet}

\usepackage[spanish]{babel}
\usepackage[utf8]{inputenc}
\usepackage{physics}
\usepackage{amsmath}
\usepackage{bookmark}
\usepackage{amsfonts}
\usepackage{amssymb}
\usepackage{mathtools}
\usepackage{graphicx}
\usepackage{float}
\usepackage[rightcaption]{sidecap}

%addtolength{\oddsidemargin}{.875in} addtolength{\evensidemargin}{.875in} addtolength{\textwidth}{1.75in}
%addtolength{\topmargin}{.875in} \addtolength{\textheight}{1.75in}

\title{Physical biology}
\author{David Caro}
\date{08-01-2024}

\pdfinfo{%
  /Title    (Physical biology)
  /Author   (David Caro)
  /Creator  (David Caro)
  /Producer (David Caro)
  /Subject  (Physics)
  /Keywords ()
}

\begin{document}
\maketitle

%%%%%%%%%%%%%%%%%%%%%%%%%%%%%%%%%%%%%%%%%%%%%%%%%%%%%%%%%%%%%%%%%%%%%%%%%
\section{1.1 Life}
Common description uses 5 characteristics:
\begin{itemize}
  \item made of cells
  \item they replicate
  \item they evolve
  \item store information (genes)
  \item use energy
\end{itemize}

\hfill\\
\section{1.2 Cellular theory}
\begin{itemize}
  \item 1665 Hook -> first microscope -> \textbf{dead cells}\\
  \item 1660-1680 Van Leeuwenhoek -> more potent microscopes -> \textbf{live cells and microorganisms}\\
  \item 1831 Brown -> \textbf{defines the nucleus}\\
  \item 1838 Schleiden (for plants), 1839 Schwann (for animals) and 1857 Virchow define the cellular theory:
        \begin{itemize}
          \item The cell is the unit of structure for life
          \item Cells retain a dual existence as individuals and building blocks
          \item (Virchow 1857) All cells come from other cells
        \end{itemize}
\end{itemize}

\hfill\\
\section{1.3 Theory of evolution}
Darwin and Wallace create the theory of evolution, two main principles:
\begin{itemize}
  \item All species are related by common ancestors.
  \item Characteristics of species change from generation to generation.
\end{itemize}
The key insight was their description of the process that pushes for that change: \textbf{natural selection}.\\
This means that you can draw a \textbf{tree of life} from the common ancestor to the current extant species.

\hfill\\
\section{1.4 Chromosomic theory of inheritance and central dogma}
Chromosomes are made of a single DNA molecule, and some of it's segments that codify the products in the cell are called
genes.\\
The central dogma of microbiology states that the flow of information is unidirectional:
\begin{itemize}
  \item DNA
  \item -transcription-> mRNA
  \item -traduction-> protein
  \item -> specific trait
\end{itemize}

\hfill\\
\section{1.5 Taxonomy}
Naming organisms, started by Carl Linnaeus, binomial system, ex:\\
\begin{center}
  <gender> <species>: quercus robur (oak)
\end{center}
Added a hierarchy of taxonomical groups:\\
\begin{center}
  species < gender < family < order < class < phylum < kingdom
\end{center}
Can be drawn for all species in a phylogenetic tree, where the closest the branch, the more closely related the
species.\\
Recent genetic studies have shown that this is obsolete, and currently life is classified in three domains:
\begin{itemize}
  \item Bacteria
  \item Archaea
  \item Eukarya (cells with well defined nucleus, plants, fungi, animals, ...)
\end{itemize}

\section{2.0 Biomolecules}
There's organic and inorganic molecules that are part of a living being, we will focus on the organic ones:

\hfill\\
\section*{2.1 Proteins}
Polymers of aminoacids joined by peptide bonds, structure of an aminoacid:
\includegraphics{images/amino_acid_structure}\\
{\footnotesize by Smokefoot - Own work, Public Domain, https://commons.wikimedia.org/w/index.php?curid=106539890}\\
Note the amino group $NH_2$, the carboxyl acid group $COOH$, and the lateral chain with the root $R$, characteristic of
every aminoacid.\\
They are joined by condensation, when the $COOH$ group creates a peptidic bind with the $NH2$ of the next.

\hfill\\
\section*{2.2 Protein structure}
There's 4 structure levels:
\begin{itemize}
  \item Primary: peptidic bonds between single aminoacids in the protein
  \item Secondary: hydrogen bonds between the $O$ of a $COOH$ group in one aminoacid and the $NH_2$ of another, can
        create two different shapes:
        \begin{itemize}
          \item $\alpha$-helix - $R$ groups facing outwards
          \item $\beta$-sheet
        \end{itemize}
        \includegraphics[width=0.2\textwidth]{images/secondary_protein_structure.png}
  \item Tertiary: \textbf{when $R$ groups are involved}, there's many kind of folds, but only a few bonds that can
        happen:
        \begin{itemize}
          \item \textbf{Hydrogen bonds} between $COOH$ carbonyl group and the lateral chain
          \item \textbf{Hydrogen bonds} between two lateral chains or $R$ groups
          \item \textbf{Covalent bonds}, commonly di-sulfur bridge between cysteine $R$ groups
          \item \textbf{Ionic bonds} between $R$ groups
          \item \textbf{Hydrophobe interactions and van der Waals forces}, when in water, the hydrophile lateral chains
                push the hydrophobe $R$ groups together, and then van der Waal forces keep them stable
        \end{itemize}
  \item Quaternary: Combination of polypeptide, bound by similar bonds than the tertiary structures.
\end{itemize}
Folding is often facilitated by a specific type of proteins called \textbf{chaperones}. These molecules are generated in
big quantities when there's a high rise in temperature. They attach themselves to the hydrophobe sections of unfolded
proteins to prevent other molecules from attaching and allow the protein to re-fold itself before any unfolded
aggregates get created.


\hfill\\
\section*{2.3 Protein function}
The functionality of a protein is strongly related to it's folding, two proteins with the same aminoacid sequence cat
behave really differently, for example prions are proteins that when folded in a specific way, become infectious.\\
When a protein loses it's folding it's said it gets \textbf{denaturated}, this can happen for many reasons (heat, pH,
...).\\
They are the more versatile of the molecule groups, having many functions:\
\begin{itemize}
  \item \textbf{Catalytic/enzymes}: they speed up many chemical reactions.
  \item \textbf{Defensive}: Antibodies and other proteins attack and destroy viruses and bacteria
  \item \textbf{Movement}: Motor protein and contractile proteins move substances within the cell, the cells themselves
        and the whole body (muscles).
  \item \textbf{Signaling}: They are involved in the transport and reception of signals, sometimes bound to the cell
        membrane to interact with neighboring cells.
  \item \textbf{Structural}: collagen of the skin and tendons, membrane proteins.
  \item \textbf{Transport}: They allow that some molecules enter and leave the cell, or transport them throughout the
        body (hemoglobin).
\end{itemize}

\hfill\\
\section*{2.4 Nucleic acids}
Formed by \textbf{nucleotides}: a pentose sugar (ribose with $OH$ on 2'/deoxyribose with $H$ on 2'), a nitrogenous base
(bound to the carbon 1'), and a phosphate (bound to the carbon 5').\\
Note that \textbf{nucleoside} is just the pentose sugar and the base.\\
Bases can be one of:
\begin{itemize}
  \item Cytosine - Pyrimidine
  \item Uracil (RNA)/Thymine (DNA) - Pyrimidine
  \item Guanine - Purine
  \item Adenine - Purine
\end{itemize}
The nucleotides are bound with phosphodiester bonds (covalent bonds) on 5' and 3', and form a directed chain, always
written from the nucleotid with the phosphate (5') free, to the one with the $OH$ (3') free.\\
That is also the direction they are synthesized.\\

They form two main structures:
\begin{itemize}
  \item \textbf{RNA}:
        \begin{itemize}
          \item Sugar: ribose
          \item Bases: A-U, G-C
          \item Structure: simple strand
          \item Function: transport, structural, etc.
        \end{itemize}
  \item \textbf{DNA}:
        \begin{itemize}
          \item Sugar: deoxyribose
          \item Bases: A-T, G-C
          \item Structure: double helix strand bound by hydrogen bonds of the bases
          \item Function: carries the genetic information
        \end{itemize}
\end{itemize}

In order to polymerize the nucleotides, the potential energy of the nucleotides is increased by adding phosphates,
creating triphosphate nucleosides or \textbf{activated nucleotides}, then when they get polymerized they need water and
free inorganic pyrophosphate. Ex. ATP (adenine + 2 phosphates -> adenosin triphosphate) \\

\hfil\\
\section*{2.5 DNA structure}


\end{document}
