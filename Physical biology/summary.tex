\documentclass[a4paper,landscape,10pt]{cheatsheet}

\usepackage[english]{babel}
\usepackage[utf8]{inputenc}
\usepackage{physics}
\usepackage{amsmath}
\usepackage{bookmark}
\usepackage{amsfonts}
\usepackage{amssymb}
\usepackage{mathtools}
\usepackage{graphicx}
\usepackage{float}
\usepackage[rightcaption]{sidecap}

%addtolength{\oddsidemargin}{.875in} addtolength{\evensidemargin}{.875in} addtolength{\textwidth}{1.75in}
%addtolength{\topmargin}{.875in} \addtolength{\textheight}{1.75in}

\title{Physical biology}
\author{David Caro}
\date{08-01-2024}

\pdfinfo{%
  /Title    (Physical biology) /Author   (David Caro) /Creator  (David Caro) /Producer (David Caro) /Subject  (Physics)
  /Keywords () }

\begin{document}
\maketitle

%%%%%%%%%%%%%%%%%%%%%%%%%%%%%%%%%%%%%%%%%%%%%%%%%%%%%%%%%%%%%%%%%%%%%%%%%
\section{1.1 Life}
Common description uses 5 characteristics:
\begin{itemize}
      \item made of cells
      \item they replicate
      \item they evolve
      \item store information (genes)
      \item use energy
\end{itemize}

\hfill\\
\section{1.2 Cellular theory}
\begin{itemize}
      \item 1665 Hook -> first microscope -> \textbf{dead cells}\\
      \item 1660-1680 Van Leeuwenhoek -> more potent microscopes -> \textbf{live cells and microorganisms}\\
      \item 1831 Brown -> \textbf{defines the nucleus}\\
      \item 1838 Schleiden (for plants), 1839 Schwann (for animals) and 1857 Virchow define the cellular theory:
            \begin{itemize}
                  \item The cell is the unit of structure for life
                  \item Cells retain a dual existence as individuals and building blocks
                  \item (Virchow 1857) All cells come from other cells
            \end{itemize}
\end{itemize}

\hfill\\
\section{1.3 Theory of evolution}
Darwin and Wallace create the theory of evolution, two main principles:
\begin{itemize}
      \item All species are related by common ancestors.
      \item Characteristics of species change from generation to generation.
\end{itemize}
The key insight was their description of the process that pushes for that change: \textbf{natural selection}.\\
This means that you can draw a \textbf{tree of life} from the common ancestor to the current extant species.

\hfill\\
\section{1.4 Chromosomic theory of inheritance and central dogma}
Chromosomes are made of a single DNA molecule, and some of it's segments that codify the products in the cell are called
genes.\\
The central dogma of microbiology states that the flow of information is unidirectional:
\begin{itemize}
      \item DNA
      \item -transcription-> mRNA
      \item -traduction-> protein
      \item -> specific trait
\end{itemize}

\hfill\\
\section{1.5 Taxonomy}
Naming organisms, started by Carl Linnaeus, binomial system, ex:\\
\begin{center}
      <gender> <species>: quercus robur (oak)
\end{center}
Added a hierarchy of taxonomical groups:\\
\begin{center}
      species < gender < family < order < class < phylum < kingdom
\end{center}
Can be drawn for all species in a phylogenetic tree, where the closest the branch, the more closely related the
species.\\
Recent genetic studies have shown that this is obsolete, and currently life is classified in three domains:
\begin{itemize}
      \item Bacteria
      \item Archaea
      \item Eukarya (cells with well defined nucleus, plants, fungi, animals, ...)
\end{itemize}

\section{2.0 Biomolecules}
There's organic and inorganic molecules that are part of a living being, we will focus on the organic ones:

\hfill\\
\section*{2.1 Proteins}
Polymers of aminoacids joined by peptide bonds, structure of an aminoacid:
\includegraphics{images/amino_acid_structure}\\
{\footnotesize by Smokefoot - Own work, Public Domain, https://commons.wikimedia.org/w/index.php?curid=106539890}\\
Note the amino group $NH_2$, the carboxyl acid group $COOH$, and the lateral chain with the root $R$, characteristic of
every aminoacid.\\
They are joined by condensation, when the $COOH$ group creates a peptidic bind with the $NH2$ of the next.

\hfill\\
\section*{2.2 Protein structure}
There's 4 structure levels:
\begin{itemize}
      \item Primary: peptidic bonds between single aminoacids in the protein
      \item Secondary: hydrogen bonds between the $O$ of a $COOH$ group in one aminoacid and the $NH_2$ of another, can
            create two different shapes:
            \begin{itemize}
                  \item $\alpha$-helix - $R$ groups facing outwards
                  \item $\beta$-sheet
            \end{itemize}
            \includegraphics[width=0.2\textwidth]{images/secondary_protein_structure.png}
      \item Tertiary: \textbf{when $R$ groups are involved}, there's many kind of folds, but only a few bonds that can
            happen:
            \begin{itemize}
                  \item \textbf{Hydrogen bonds} between $COOH$ carbonyl group and the lateral chain
                  \item \textbf{Hydrogen bonds} between two lateral chains or $R$ groups
                  \item \textbf{Covalent bonds}, commonly di-sulfur bridge between cysteine $R$ groups
                  \item \textbf{Ionic bonds} between $R$ groups
                  \item \textbf{Hydrophobe interactions and van der Waals forces}, when in water, the hydrophile lateral chains
                        push the hydrophobe $R$ groups together, and then van der Waal forces keep them stable
            \end{itemize}
      \item Quaternary: Combination of polypeptide, bound by similar bonds than the tertiary structures.
\end{itemize}
Folding is often facilitated by a specific type of proteins called \textbf{chaperones}. These molecules are generated in
big quantities when there's a high rise in temperature. They attach themselves to the hydrophobe sections of unfolded
proteins to prevent other molecules from attaching and allow the protein to re-fold itself before any unfolded
aggregates get created.


\hfill\\
\section*{2.3 Protein function}
The functionality of a protein is strongly related to it's folding, two proteins with the same aminoacid sequence cat
behave really differently, for example prions are proteins that when folded in a specific way, become infectious.\\
When a protein loses it's folding it's said it gets \textbf{denaturated}, this can happen for many reasons (heat, pH,
...).\\
They are the more versatile of the molecule groups, having many functions:\
\begin{itemize}
      \item \textbf{Catalytic/enzymes}: they speed up many chemical reactions.
      \item \textbf{Defensive}: Antibodies and other proteins attack and destroy viruses and bacteria
      \item \textbf{Movement}: Motor protein and contractile proteins move substances within the cell, the cells themselves
            and the whole body (muscles).
      \item \textbf{Signaling}: They are involved in the transport and reception of signals, sometimes bound to the cell
            membrane to interact with neighboring cells.
      \item \textbf{Structural}: collagen of the skin and tendons, membrane proteins.
      \item \textbf{Transport}: They allow that some molecules enter and leave the cell, or transport them throughout the
            body (hemoglobin).
\end{itemize}

\hfill\\
\section*{2.4 Nucleic acids}
Formed by \textbf{nucleotides}: a pentose sugar (ribose with $OH$ on 2'/deoxyribose with $H$ on 2'), a nitrogenous base
(bound to the carbon 1'), and a phosphate (bound to the carbon 5').\\
Note that \textbf{nucleoside} is just the pentose sugar and the base.\\
Bases can be one of:
\begin{itemize}
      \item Cytosine - Pyrimidine
      \item Uracil (RNA)/Thymine (DNA) - Pyrimidine
      \item Guanine - Purine
      \item Adenine - Purine
\end{itemize}
The nucleotides are bound with phosphodiester bonds (covalent bonds) on 5' and 3', and form a directed chain, always
written from the nucleotid with the phosphate (5') free, to the one with the $OH$ (3') free.\\
That is also the direction they are synthesized.\\

They form two main structures:
\begin{itemize}
      \item \textbf{RNA}:
            \begin{itemize}
                  \item Sugar: ribose
                  \item Bases: A-U, G-C
                  \item Structure: simple strand
                  \item Function: transport, structural, etc.
            \end{itemize}
      \item \textbf{DNA}:
            \begin{itemize}
                  \item Sugar: deoxyribose
                  \item Bases: A-T, G-C
                  \item Structure: double helix strand bound by hydrogen bonds of the bases
                  \item Function: carries the genetic information
            \end{itemize}
\end{itemize}

In order to polymerize the nucleotides, the potential energy of the nucleotides is increased by adding phosphates,
creating triphosphate nucleosides or \textbf{activated nucleotides}, then when they get polymerized they need water and
generate inorganic pyrophosphate. Ex. ATP (adenine + 2 phosphates -> adenosin triphosphate) \\

\hfil\\
\section*{2.5 DNA structure}
The \textbf{primary structure} of the nucleic acids is just the sequence of basis it's made of.\\
\hfill\\
The \textbf{secondary structure} is built by hydrogen bonds between the $R$ groups, and for DNA is an
\textbf{antiprallel double helix}, where the bases are matched only between complementary purines and pyrimidines (G-C
has 3 hydrogen bonds, A-T has two hydrogen bonds).\\
The helix (2 nm wide) does a turn (3.4 nm) every 10 base pairs (0.34 nm), where the two strands phase is not symmetric,
creating one small and one big gap between them.\\
\hfill\\
The \textbf{tertiary structure} DNA is usually packed with proteins, in eucaryotes these are histones, the bundle is
called \textbf{chromatin}, and it comes in two flavors, a more packed one that is not ready for transcription
\textbf{heterochromatin} and a lightly packed one ready for transcription, \textbf{euchromatin}.\\

\hfil\\
\section*{2.6 RNA structure}
The \textbf{primary structure}: same as DNA, it's the sequence of basis.\\
\hfill\\
The \textbf{secondary structure}: the most common is the stem loops (horquilla), where a single strand bends over itself
to create a loop and a double helix section with itself.\\
\hfill\\
The \textbf{tertiary structure}: secondary structures fold to generate a great variety of forms for RNA.\\

\hfil\\
\section*{2.7 RNA types}
There's three main types of RNA:\\
\begin{itemize}
      \item \textbf{mRNA (messenger RNA)}:
            \begin{itemize}
                  \item Transports information from the nucleus to the cytoplasm
                  \item Gets translated in the ribosomes to generate proteins
                  \item Single strand \end{itemize}TOTAL   26 TiB   21 TiB   21 TiB  1.4 GiB   67 GiB  5.2 TiB  79.97 15:43:15
            TOTAL   26 TiB   21 TiB   21 TiB  1.4 GiB   67 GiB  5.3 T

      \item \textbf{rRNA (ribosomal RNA)}:
            \begin{itemize}
                  \item It's a structural part of the ribosomes
                  \item There's several types that form the small and big subunits of the ribosome along with proteins
                  \item Single strand with secondary structure
            \end{itemize}

      \item \textbf{tRNA (transfer RNA)}:
            \begin{itemize}
                  \item Transports aminoacids to the ribosome
                  \item Simple strand with clover-like secondary structure
            \end{itemize}
\end{itemize}


\hfil\\
\section*{2.8 Carbohydrates}
Made out of \textbf{monosaccharides}, they are composed by carbon, hydrogen and oxygen and are mainly used either as
energy storage, or as structural molecules.\\
They have one carbonyl group ($C=O$), several hydroxyl ($-OH$) and a variable number of carbon-hydrogen bonds ($C-H$).\\
\begin{itemize}
      \item \textbf{monosaccharides}: simplest of sugars, made of a single strand of carbons, with a carbonyl group and one
            or more hydroxyl (ex. glucose, fructose, galactose).
      \item \textbf{disaccharides}: formed by two monosaccharides when the hydroxyl groups bond (by condensation) to form
            \textbf{O-glycosidic bonds} (ex. 2 x glucose -> maltose + $H2O$, glucose + galactose -> lactose + $H2O$)
      \item \textbf{oligosaccharides and polysaccharides}: for groups of 2-10 monosaccharides bonded by glycosidic bonds, we
            call them oligosaccharides, and when they have more monosaccharides, we call them polysaccharides. (ex. starch,
            cellulose, chitin, glycogen).
\end{itemize}

\hfill\\
\section*{2.9 Carbohydrates: polysaccharides}
They have mainly two functions, \textbf{structural} and \textbf{energy reserve}. They also have a role in cell
identification (ex. sperm attaching to the ovule). The main polysaccharides of life are:
\begin{itemize}
      \item \textbf{Starch}
            \begin{itemize}
                  \item reserve molecule used by plants
                  \item amylose and amylopectin (both made of $\alpha$-glucose)
                  \item amylose has a lineal helix structure
                  \item amylopectin has a ramified helix structure
            \end{itemize}
      \item \textbf{Glycogen}
            \begin{itemize}
                  \item reserve molecule used by animals, fungi and some bacteria
                  \item made of $\alpha$-glucose, very similar to amylopectin but it's even more ramified
            \end{itemize}
      \item \textbf{Cellulose}
            \begin{itemize}
                  \item Structural support for plants and many algae
                  \item parallel chains of $\beta$-glucose joined with hydrogen bonds
            \end{itemize}
      \item \textbf{Chitin}
            \begin{itemize}
                  \item Structural support for insects and fungi
                  \item parallel chains of NAG (N-acetylglucosamine) joined with alternating hydrogen bonds
            \end{itemize}
      \item \textbf{Peptidoglycan}
            \begin{itemize}
                  \item Structural for bacteria
                  \item parallel chains joined with peptide bonds (using a chain of 4 aminoacids from NAM)
                  \item alternating NAG and NAM, N-acetylmuramaic acid, like NAG but with 4 aminoacids on C-3
            \end{itemize}
\end{itemize}


\hfill\\
\section*{2.10 Lipids}
Main characteristics (classified by their physical attributes, not chemical structure like polysaccharides):
\begin{itemize}
      \item Insoluble in water
      \item Soluble in organic solutions (ethanol, chloroform, ether, ...)
      \item Three main elements, \textbf{carbon, hydrogen and oxigen}, smaller amount of others \textbf{nitrogen,
                  phosphorous, sulfur}.
      \item can be found bonding to other molecules with \textbf{covalent bonds (glucolipids)} or \textbf{non-covalent
                  (lipoproteins)}.
\end{itemize}

Common ones (among others):
\begin{itemize}
      \item Phospholipids
      \item Triglycerides
      \item Steroids
\end{itemize}

\section*{2.11 Steroids}
Composed of 4 fused rings (steroid rings) and an isoprene. \\
Amphiphile  molecules.\\
\medskip
Main functions:
\begin{itemize}
      \item Signaling - hormones (ex. testosterone)
      \item Structural - changing membrane fluidity (ex. cholesterol)
\end{itemize}

\section*{2.12 Triglycerides}
Composed of a glycerol molecule with 3 fatty acids attached with ester bonds (if 2 fatty acids -> diglyceride, if 1
monoglyceride)\\
\medskip
They are neutral  molecules.\\
\medskip
Can be split into:
\begin{itemize}
      \item Oils: when they have non-saturated fatty acids (liquid at room temperature), common in plants.
      \item Fats: when they only have saturated fatty acids (solid at room temperature), common in animals.
\end{itemize}
Main functions is energy reserve.


\section*{2.13 Phospholipids}
Components of \textbf{cell walls}, formed by a \textbf{glycerol} attached to \textbf{two fatty acids or isoprenoids} and
a \textbf{phosphate}, and a \textbf{variable polar molecule} attached to the phosphate.\\
\medskip
They are \textbf{anphypathics}, with the fatty acids being non-polar, and the head polar.\\
\medskip
They make structures like \textbf{liposomes} (doubled-layered sphere), \textbf{micelles} (single layered sphere) and
bilayer sheets.\\
\medskip
\textbf{Archaea} use isoprenoids and \textbf{eukarya and bacteria} use fatty acids.


\section*{3 Cell structure}
\subsection*{3.1 Procaryote cells}
Procaryote cells reproduce by binary fission (no mitosis/meiosis). Are usually smaller than eukaryotes (1/10).
\begin{itemize}
      \item \textbf{Nucleoid}: contains the super-coiled chromosome (usually circular DNA + non-histone proteins), with many
            genes (chunks that contain RNA building information)
      \item \textbf{Plasmids}: small circular strands of supercoiled DNA independent from the nucleoid, also contain genes.
      \item \textbf{Ribosomes}: Generate proteins from RNA strands, made up of a small and a large subunit.
      \item \textbf{Cytoskeleton}: Structures the cell interior and it's shape, and take a main role in cell division.
      \item \textbf{Membrane complexes}: Some photosynthetic species commonly have infoldings of the cellular membrane that
            contain enzymes and pigment molecules.
      \item \textbf{Cell wall}: To resist the osmosis pressure, most bacteria and archaea have a stiff cell wall. In most
            bacteria the cell wall main component is the modified polysaccharide peptiodglycan. While archaea use many other
            substances.
      \item \textbf{Specialized organelles}: Some species have "organelles" that contain enzymes, or other structures.
      \item \textbf{Flagellum}: Made to move through water, the common building block between archaea and bacteria is the
            rotating motor at the base of the flagellum where it joins the membrane.
      \item \textbf{Fimbriae}: smaller than the flagellum and without motor, usually more numerous allows bacteria to glue
            themselves to other cells or surfaces.
\end{itemize}

\section*{3.2 Eukaryote cells}
Defined as having a well differentiated nucleus.
\section*{3.2.1 Common organelles}
\begin{itemize}
      \item \textbf{Nucleus}: Double membrane formed by the \textbf{nuclear envelope} and the \textbf{lamina} lattice that
            gives it shape. It contains the densely packed chromatin (chromosomes with histamines, \textbf{heterochromatin})
            in the exterior and the less densely packed ones in the interior (\textbf{euchromatin}), with one central
            tightly packed region called \textbf{nucleolus}. The nucleus synthesizes RNA, and also proteins from RNA, for
            example the \textbf{nucleolus} produces the large and small ribosomal units. It has many \textbf{nuclear pores}
            that allow the traffic of proteins and other molecules from inside and outside the nucleus.
      \item \textbf{Rough Endoplasmic Reticulum (endomembrane system)}: It's the part of the ER closest to the nucleus, that
            has many ribosomes attached to it, forming a network of sacs and flattened tubules. The ribosomes on it's
            surface produce proteins that get into the lumen and get folded and processed and will either stay in the ER to
            be used or packaged in vesicles and sent to other destinations (other organelles, membrane, even to other
            cells).
      \item \textbf{Smooth Endoplasmic Reticulum (endomembrane system)}: This is the part of the ER that has no ribosomes
            attached. It contains enzymes that can synthesize lipids, or modify them and other toxic molecules. It also
            serves as reservoire of $Ca^+$.
      \item \textbf{Golgi apparatus (endomembrane system)}: Most of the proteins generated by the RER pass through the
            \textbf{cis} side of the Golgi apparatus, where they are merged with the \textbf{cisterna}, and pass from
            cisterna to cisterna where they react with the different enzymes on each, towards the \textbf{trans} side where
            they get packaged in vesicles and sent to their destinations.
      \item \textbf{Peroxisomes}: Generated when empty vesicles from the ER get filled with peroxisome-specific enzymes from
            teh cytosol. They are centers for redox reactions, that usually generate hydrogen peroxide as a side-product, so
            the peroxisome has specific enzymes to catalyze the reaction to transform it into water and oxygen.
      \item \textbf{Mitochondria}: In charge of the creation of ATP from carbohydrates, lipids and proteins. With a double
            bilayer membrane (this is 4x layers). The inner membrane forms \textbf{cristae} sacs, and contains the
            \textbf{mitochondrial matricx} (fluid inside). They are highly dynamic in shape. Each has many copies of a
            usually circular chromosome called \textbf{mitochondrial DNA (mtDNA)}, that contains the genes for
            \textbf{mitochondrial ribosomes} among others. These ribosomes are smaller than the ones in the cytosol and
            produce some of the mitochondria proteins.
      \item \textbf{Cytoskeleton}: Extensive system of protein fibers, split in three main subsystems:
            \begin{itemize}
                  \item \textbf{Actin filaments}: Made of actin, they:
                        \begin{itemize}
                              \item maintain cell shape by resisting tension (pull)
                              \item move cells via muscle contraction or cell crawling
                              \item divide animal cells in two
                              \item move organelles and cytoplasm in plants, fungi and animals
                        \end{itemize}
                  \item \textbf{Intermediate filaments}: Made of several different proteins, they:
                        \begin{itemize}
                              \item maintain cell shape by resisting tension (pull)
                              \item anchor nucleus and some other organelles (ex. nuclear lamina)
                        \end{itemize}
                  \item \textbf{Microtubules}: Made of $\alpha-$ and $\beta-tubulin$ dimers forming an empty tubule. They
                        originate from the \textbf{microtubule organizing centre (MTOC)}. Plants usually have many, animals and
                        fungi usually have only one, called \textbf{centrosome}, in animals is composed by two elements called
                        \textbf{centrioles}. They have several functions:
                        \begin{itemize}
                              \item maintain cell shape by resisting compression (push)
                              \item move cells via flagella or cilia
                              \item move chromosomes during cell division
                              \item assist formation of cell plate during plant cell division
                              \item provide tracks for intracellular transport (through kinesin)
                        \end{itemize}
            \end{itemize}
      \item \textbf{Plasma membrane}: Bi-layer membrane made mainly of Phospholipids, with proteins embedded on one side or
            through the membrane, they control the traffic of substances from inside and outside the cell. There's three
            types of transport:
            \begin{itemize}
                  \item Osmosis (no extra energy required): Movement of water through the membrane from the side with less
                        solute to the side with more (diffusion of water).
                  \item Passive transport (no extra energy required)
                        \begin{itemize}
                              \item Simple diffusion: Movement of non-polar substances through the membrane (like osmosis for
                                    non-water molecules, ex. $O_2$, $CO_2$).
                              \item Facilitated diffusion: Movement of water soluble substances through specialized canal proteins
                                    embedded in the membrane. Might also happen with proteins that join the substance to pass the
                                    membrane.
                        \end{itemize}
                  \item Active transport (requires energy, ATP): It goes against the concentration gradient, or with substances
                        that can't pass the membrane through diffusion. It uses transport proteins, and might involve the
                        co-transport of substances, either the same direction (\textbf{synporter}) or the opposite
                        (\textbf{antiporter}).

            \end{itemize}
\end{itemize}
\section*{3.2.2 Animal only}
\begin{itemize}
      \item \textbf{Lysosomes (endomembrane system)}: Generated from the Golgi apparatus, contain enzymes specialized on the hydrolyzation of different
            molecules (proteins, nucleic acids, lipids and carbohydrates). They maintain an acid interior by using ion pumps on
            their membrane so the hydrolases keep being effective.
      \item \textbf{Centrioles}: Part of the centrosome of the animal cells, formed by two elements and a diffuse matrix
            of proteins where all the microtubules get formed, they are an important part of cell division and get
            themselves duplicated in it's early stages.
      \item \textbf{Extra cellular matrix}: Similar to the cell wall of plants, but more diffuse mixture of secreted
            proteins and polysaccharides that often support the animal cell.
\end{itemize}

\section*{3.2.3 Plant only}
\begin{itemize}
      \item \textbf{Vacuoles}:
      \item \textbf{Chloroplasts}: They have a double membrane like mitochondria, and their own DNA and ribosomes.
            Unlike mitochondria though, the inner membrane does not wrap to form crests, but instead a third membrane forms
            sac-like \textbf{thylakoids} that are stacked into connected \textbf{grana}. They have pigments like
            chlorophyll and carotene in their \textbf{stroma} (fluid filled space inside the inner membrane) that
            convert light energy into chemical energy that then the enzymes use to produce sugars.
      \item \textbf{Cell wall}: Made out of cellulose, it wraps the plant cell and gives it mechanical strength. It also
            limits the volume.
\end{itemize}

\end{document}
