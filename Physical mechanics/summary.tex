\documentclass[a4paper,landscape,10pt]{cheatsheet}

\usepackage[spanish]{babel}
\usepackage[utf8]{inputenc}
\usepackage{physics}
\usepackage{amsmath}
\usepackage{bookmark}
\usepackage{amsfonts}
\usepackage{amssymb}
\usepackage{mathtools}
\usepackage{graphicx}
\usepackage{float}
\usepackage[rightcaption]{sidecap}

%addtolength{\oddsidemargin}{.875in} addtolength{\evensidemargin}{.875in} addtolength{\textwidth}{1.75in}
%addtolength{\topmargin}{.875in} \addtolength{\textheight}{1.75in}

\title{Physical mechanics}
\author{David Caro}
\date{09-01-2024}

\pdfinfo{%
  /Title    (Physical mechanics) /Author   (David Caro) /Creator  (David Caro) /Producer (David Caro) /Subject
  (Physics) /Keywords () }

\begin{document}
\maketitle

%%%%%%%%%%%%%%%%%%%%%%%%%%%%%%%%%%%%%%%%%%%%%%%%%%%%%%%%%%%%%%%%%%%%%%%%%
\section{1 Cinematics}
Having:
\begin{itemize}
  \item $\mathcal{R}$ as rotation matrix of $S'$ seen from $S$
  \item $\bf{R}$ the relative position of $S'$ seen from $S$
\end{itemize}
the space coordinates of a point are:
\begin{gather*}
  r = \mathcal{R}^T r' + \bf{R} \\
  r'= \mathcal{R}(r - \bf{R})
\end{gather*}

We can then define the antisymmetric \textbf{angular momentum matrices} as:
\begin{gather*}
  \hat{w} = \mathcal{\dot{R}}^T\mathcal{R} = \begin{pmatrix}
    0    & -w_z & w_y  \\
    w_z  & 0    & -w_x \\
    -w_y & w_x  & 0
  \end{pmatrix} \\
  \hat{w}' = \mathcal{R}\mathcal{\dot{R}}^T = \begin{pmatrix}
    0     & -w'_z & w'_y  \\
    w'_z  & 0     & -w'_x \\
    -w'_y & w'_x  & 0
  \end{pmatrix}
\end{gather*}

And the a can then define the dual vectors too:
\begin{gather*}
  w = (w_x, w_y, w_z)\\
  w' = (w'_x, w'_y, w'_z)\\
\end{gather*}

We get the relationships:
\begin{gather*}
  \hat{w}q = w \cross q \\
  \hat{w}'q = w' \cross q \\
  \hat{w} = \mathcal{R}^T\hat{w}'\mathcal{R} \\
  \hat{w}' = \mathcal{R}\hat{w}
\end{gather*}

Defining $\textbf{V}$ as the speed of $S'$ as seen from $S$, we have the velocities (deriving the position one):
\begin{gather*}
  v=\mathcal{R}^T(v'+w'\cross r') + \textbf{V} \\
  v'=\mathcal{R}[v-\textbf{V}-w\cross(r-\textbf{R})]
\end{gather*}

Then with $\alpha$ as the angular acceleration, $A$ as the acceleration of $S'$ as seen by $S$, and deriving this once
more we get the acceleration:
\begin{gather*}
  a' = -\alpha'\cross r' \text{; acimutal}\\
  \quad\quad - 2w'\cross v' \text{; coriolis}\\
  \quad\quad - w'\cross (w'\cross r') \text{; centrifugal}\\
  \quad\quad + \mathcal{R}a \\
  \quad\quad -\mathcal{R}\textbf{A} \text{; drag}\\
  \\
  a = \textbf{A} + \mathcal{R}^T[a' \\
  \quad\quad +\alpha'\cross r' \\
  \quad\quad + 1w'\cross v' \\
  \quad\quad + w'\cross (w'\cross r')] \\
\end{gather*}


\section{2 Dynamics}
\subsection*{2.1 Newton dynamics}
The three newton laws:
\begin{itemize}
  \item In absence of forces, a body remains it's constant lineal movement $\bf{p} = m\bf{v}$:
        $$
          \bf{0} = \derivative{}{t}\bf{p} \longrightarrow \bf{p} = \text{ct.}
        $$
  \item The change of lineal momentum is given by the force that acts on a body:
        $$
          \bf{F} = \derivative{}{t}\bf{p}; \quad m=\text{ct.} \rightarrow \bf{F} = m\bf{a}
        $$
  \item Given two particles that interact, the force on particle 1 is the same strength and opposite sign than the force
        on particle 2.
        $$
          \bf{F}_1 = -\bf{F}_2
        $$
\end{itemize}

\subsection*{2.2 Common differential equations and solutions}
Common types of problems and their differential equations and solutions:
\begin{itemize}
  \item Uniform rectilinear movement:
        $$
          \ddot{q} = 0 \longrightarrow q(t) = q_0 + \dot{q}_0 t
        $$
  \item Uniformly accelerated movement:
        $$
          \ddot{q} = a \longrightarrow q(t) = q_0 + \dot{q}_0 t + \frac{1}{2}at^2
        $$
  \item Uniformly accelerated with friction:
        $$
          \ddot{q} = a - b\dot{q} \longrightarrow \dot{q}(t) = \frac{a}{b} + \left(\dot{q}_0 - \frac{a}{b}\right)e^{-bt}
        $$
        Where $a/b$ is the limit speed.
  \item Harmonic oscillator:
        \begin{gather*}
          \ddot{q} + \omega^2 q=0 \rightarrow                             \\
          q(t)=\frac{\dot{q}_0}{\omega}\sin(\omega t) + q_0\cos(\omega t) \\
          \quad = A\sin(\omega t+\phi)
        \end{gather*}
        Where
        \begin{gather*}
          A=\sqrt{q^2_0 + \frac{\dot{q}^2_0}{\omega^2}} \\
          \phi = \arccos\frac{\dot{q}_0}{\sqrt{q^2_0\omega^2+\dot{q}^2_0}}
        \end{gather*}
  \item Charged particle in an electric and magnetic field (Lorentz force):
        \begin{gather*}
          \bf{F} = q\bf{E}+q\bf{v}\cross\bf{B} \\
        \end{gather*}
        Ends up creating a spiral with the axes perpendicular to the magnetic field, that turns with a frequency $\omega_c$
        called \textbf{cyclotronic frequency}, and a constant modulo of the speed.
\end{itemize}

\subsection*{2.3 Non-inertial reference frames}
When applying newton dynamics from the point of view of a non-inertial reference frame, we get fictional forces:
\begin{gather*}
  m\bf{a}' = \bf{F}' \quad ;\mathcal{R}\bf{F}\\
  \qquad -m\mathcal{R}\bf{A} \quad ;\bf{F}'_\text{drag}\\
  \qquad -m\boldsymbol{\dot{\omega}}'\cross \bf{r}' \quad ;\bf{F}'_\text{acimutal}\\
  \qquad - 2m\boldsymbol{\omega}'\cross\bf{v}' \quad ;\bf{F}'_{\text{Coriolis}}\\
  \qquad -m\boldsymbol{\omega}'\cross(\boldsymbol{\omega}'\cross\bf{r}')  \quad ;\bf{F}'_{\text{centrifugal}}
\end{gather*}

Then, for a particle on the surface of earth we have:
\begin{gather*}
  \bf{a}' = \frac{1}{m}\mathcal{R}\bf{F}_\text{ng} + \textbf{g}'_\text{effective} - 2\boldsymbol{\omega}'\cross \textbf{v}' \\
\end{gather*}
where $\bf{F}_\text{ng}$ is the non-gravitational force and the effective gravitational acceleration is:
\begin{gather*}
  \bf{g}'_\text{effective} = (-\omega^2_T R_T \cos\lambda\sen\lambda)\textbf{e}'_y \\
  \quad + (-g + \omega^2_T R_T \cos^2\lambda)\bf{e}'_z
\end{gather*}
Where $\lambda$ is the latitude.


\section*{3 Geometry of particle systems}
\subsection*{3.1 Center of mass}
\begin{gather*}
  \textbf{R}_{CM} \equiv \frac{1}{M_{\text{total}}}\sum_{i=1}^{N}m_i\textbf{r}_i \\
  \textbf{R}_{CM} \equiv \frac{1}{M_{\text{total}}}\int_{\mathcal{V}}\rho(\textbf{r})\textbf{r}d\mathcal{V}
\end{gather*}
Note that for non-euclidean coordinates we have different $d\mathcal{V}$, so the volume integrals have extra members
(generic volume integrals):
\begin{itemize}
  \item Cylindrical (for cylinder of radius R and height Z):
        $$
          \quad \int_{0}^{R}\int_{0}^{Z}\int_{0}^{2\pi}r \quad d\Theta dz dr
        $$
  \item Spherical, for sphere of radius R:
        $$
          \quad \int_{0}^{R}\int_{0}^{2\pi}\int_{0}^{\pi}r^2\sin{\varphi} \quad d\varphi d\Theta dr
        $$
\end{itemize}

\subsection*{3.2 Inertia tensor}
Symmetric tensor, defined as:

\begin{gather*}
  \textbf{I} \equiv \sum_{i=1}^{N} m_i\textbf{r}_i^2\mathcal{I} - \sum_{i}^{N}m_i\textbf{r}_i\textbf{r}_i^T\\
\end{gather*}

In matrix form (if integrating, replace sum by volume/surface integral):
$$
  \sum_{i=1}^{N}m_i
  \begin{pmatrix*}
    y_i^2 + z_i^2 & -x_i y_i & -x_i z_i \\
    -x_i y_i & x_i^2 + z_i^2 & -y_i z_i \\
    -x_i z_i & -y_i z_i & x_i^2 + y_i^2 \\
  \end{pmatrix*}
$$

And coordinate integral form (as before, keep in mind the coordinates for the integral):
\begin{gather*}
  I_{\alpha\beta} = \int_{\mathcal{V}}\rho(\textbf{r})[\delta_{\alpha\beta}r^2-r^\alpha r^\beta]d\mathcal{V}
\end{gather*}

$I_{xx}$, $I_{yy}$ and $I_{zz}$ are called \textbf{moments of inertia with respect to the axis}. While the non-diagonal
terms are called \textbf{products of inertia}.

For \textbf{flat surfaces in the XY plane}, we have the \textbf{theorem of perpendicular axes}:
$$
  I_{zz} = I_{xx} + I_{yy}
$$



\subsection*{3.3 Inertia tensor in different reference systems}
We have the general formula for the inertia tensor in the reference system $S'$:
\begin{gather*}
  \textbf{I}' = \mathcal{R}[ \\
    \qquad \textbf{I} - \textbf{I}_M \\
    \qquad + M(\textbf{R}-\textbf{R}_{CM})^2\mathcal{I} \\
    \qquad - M(\textbf{R}-\textbf{R}_{CM})(\textbf{R}-\textbf{R}_{CM})^T \\
  ]\mathcal{R}^T
\end{gather*}
Where we defined the \textbf{inertia tensor of the center of mass}:
\begin{gather*}
  \textbf{I}_M \equiv M\textbf{R}_{CM}^2\mathcal{I} - M\textbf{R}_{CM}\textbf{R}_{CM}^T
\end{gather*}

\subsection*{3.4 Particular cases}
\begin{itemize}
  \item When $S'\equiv S''_{CM}$ has the origin in the center of mass ($\textbf{R}=\textbf{R}_{CM}$) and
        \textbf{parallel axes to $S$}, you get the \textbf{Steiner theorem}:
        \begin{gather*}
          \textbf{I}=\textbf{I}''_{CM}+\textbf{I}_{M}
        \end{gather*}
        The inertia tensor on $S$ is equal to the inertia tensor from $S''_{CM}$ plus the inertia tensor of a single particle
        with the same mass in the center of mass.

  \item When only the origin is the same as the center of mass ($\textbf{R}=\textbf{R}_{CM}$), then we have the general
        expression:
        \begin{gather*}
          \textbf{I}'_{CM} = \mathcal{R}\textbf{I}''_{CM}\mathcal{R}^T
        \end{gather*}
        Where $S''_{CM}$ is the reference system from the previous point.

        Given the \textbf{spectral decomposition theorem}, it's always possible to find a rotation matrix $\mathcal{R}$ that $\textbf{I}'_{CM} = \textbf{I}'_D$ gets
        diagonalized. That new system is called \textbf{principal axes reference system}.
\end{itemize}

\subsection*{3.A1 Diagonalizing the inertia tensor}
\begin{itemize}
  \item Build the characteristic polynomial: $\text{det}(\textbf{I} - \lambda\mathcal{I})$
  \item Make it equal to 0 and resolve (3rd degree polynomial in worst case), that gives you the \textbf{eigenvalues}
        ($\lambda$).
  \item Get each of the \textbf{eigenvectors} ($\textbf{v}$) with $\textbf{I}\textbf{v} = \lambda\textbf{v}$ for each
        found eigenvalue $\lambda$.
  \item Now you get the rotation matrix $\mathcal{R}$ that diagonalizes the inertia tensor by doing:
        \begin{gather*}
          \mathcal{R} = \begin{pmatrix*}
            v_{1x} & v_{2x} & v_{3x} \\
            v_{1y} & v_{2y} & v_{3y} \\
            v_{1z} & v_{2z} & v_{3z} \\
          \end{pmatrix*}
        \end{gather*}
  \item And finally get the inertia tensor by applying $\textbf{I}_D = \mathcal{R}\textbf{I}\mathcal{R}^T$
\end{itemize}

\section*{4. Dynamics of particle systems}
\subsection*{4.1. Movement of the center of mass}
From the center of mass of a system of particles, all the internal forces all cancel out, leaving only the external forces as the ones affecting
the movement of the center of mass, so we have:
\begin{gather*}
  F^{ext} = M \frac{d}{dt}\textbf{V}_{CM} \\
  \textbf{V}_{CM} = \frac{d}{dt}\textbf{R}_{CM} = \frac{1}{M}\sum_{i=1}^{N}m_i\textbf{v}_i
\end{gather*}
\subsection*{4.2. Conservation of linear momentum}
Given the linear momentum:
$$
\textbf{P}=\sum_{i=1}^{N}m_i\textbf{v}_i \equiv M\textbf{V}_{CM}
$$
It's change depends only on the external forces (it is conserved if there's no external forces):
$$
\frac{d}{dt}\textbf{P}=\sum\textbf{F}^{ext}
$$
\subsection*{4.3. Conservation of angular momentum}
Given the angular momentum:
$$
\textbf{L} \equiv \sum_{i=1}^{N}\textbf{r}_i\cross\textbf{p}_i
$$
It's change depends only on the torque:
$$
\frac{d}{dt}\textbf{L}=\sum_{i=1}^{N}\textbf{r}_i\cross\textbf{F}_i\equiv\textbf{N}
$$
If all the internal forces of the system are central forces, then the angular depends only on the external torque (if
there's no external torque, then it's conserved):
$$
\frac{d}{dt}\textbf{L}=\textbf{N}^{ext}
$$
\subsection*{4.4. Conservation of energy}
\textbf{Kinetic energy} is:
$$
T(t)\equiv\sum_{i=1}^{N}\frac{1}{2}m_i\textbf{v}^2_i(t)
$$
And it's variation is the \textbf{total potency}:
$$
P(t) \equiv \frac{dT}{dt}(t) = \sum_{i=1}^{N}\textbf{F}_i\cdot\textbf{v}_i(t)
$$
\textbf{Work} between times $t_1$ and $t_2$ is defined then as:
$$
W_(t_1\rightarrow t_2) \equiv \int_{t_1}^{t_2}P(t)dt = \int_{t_1}^{t_2}\textbf{F}(t)\frac{d\textbf{r}}{dt}(t)dt
$$
If all the forces that act on the particles are \textbf{conservative} forces, then we can define \textbf{potential energy} and work can also be defined as:
$$
W_{t_1\rightarrow t_2} \equiv V(t_1) - V(t_2)
$$

And with this two definitions, we can define the \textbf{total energy} as:
$$
E(t) = V(t) + T(t)
$$
And this is conserved when we have a system where only \textbf{conservative forces} act (the ones derived from a
potential).

\subsection*{4.5. Collisions}
There's two main types of collisions:
\begin{itemize}
  \item Elastic: When the bodies don't stick together
  \begin{itemize}
    \item Angular momentum is conserved
    \item Linear momentum is conserved
    \item Energy is conserved
  \end{itemize}
  \item Inelastic: When the bodies get stuck together
  \begin{itemize}
    \item Angular momentum is conserved
    \item Linear momentum is conserved
    \item Energy is \textbf{not} conserved
  \end{itemize}
\end{itemize}
\subsection*{4.6. Seen from the center of mass}
\begin{gather*}
  \textbf{P}'_{CM} = 0 \\
  \textbf{L}'_{CM} = \mathcal{R}\left[\textbf{L} - \textbf{L}_M - \left(\textbf{I} - \textbf{I}_M\right)\boldsymbol{\omega}\right] \\
  T'_{CM} = T - T_M \\
  \qquad - (\textbf{L}-\textbf{L}_M)\cdot\boldsymbol{\omega} \\
  \qquad + \frac{1}{2}\boldsymbol{\omega}^T(\textbf{I} - \textbf{I}_M)\boldsymbol{\omega}
\end{gather*}
Where $*_M$ are the entities as if there was a particle of mass $M$ at the center of mass.

\section*{5. Rigid solid movement}


\end{document}
