\documentclass[a4paper,landscape,10pt]{cheatsheet}

\usepackage[spanish]{babel}
\usepackage[utf8]{inputenc}
\usepackage{physics}
\usepackage{amsmath}
\usepackage{bookmark}
\usepackage{amsfonts}
\usepackage{amssymb}
\usepackage{mathtools}
\usepackage{graphicx}
\usepackage{float}
\usepackage[rightcaption]{sidecap}

%addtolength{\oddsidemargin}{.875in} addtolength{\evensidemargin}{.875in} addtolength{\textwidth}{1.75in}
%addtolength{\topmargin}{.875in} \addtolength{\textheight}{1.75in}

\title{Physical mechanics}
\author{David Caro}
\date{09-01-2024}

\pdfinfo{%
  /Title    (Physical mechanics)
  /Author   (David Caro)
  /Creator  (David Caro)
  /Producer (David Caro)
  /Subject  (Physics)
  /Keywords ()
}

\begin{document}
\maketitle

%%%%%%%%%%%%%%%%%%%%%%%%%%%%%%%%%%%%%%%%%%%%%%%%%%%%%%%%%%%%%%%%%%%%%%%%%
\section{1 Cinematics}
Having:
\begin{itemize}
  \item $\mathcal{R}$ as rotation matrix of $S'$ seen from $S$
  \item $\bf{R}$ the relative position of $S'$ seen from $S$
\end{itemize}
the space coordinates of a point are:
\begin{gather*}
  r = \mathcal{R}^T r' + \bf{R} \\
  r'= \mathcal{R}(r - \bf{R})
\end{gather*}

We can then define the antisymmetric \textbf{angular momentum matrices} as:
\begin{gather*}
  \hat{w} = \mathcal{\dot{R}}^T\mathcal{R} = \begin{pmatrix}
    0    & -w_z & w_y  \\
    w_z  & 0    & -w_x \\
    -w_y & w_x  & 0
  \end{pmatrix} \\
  \hat{w}' = \mathcal{R}\mathcal{\dot{R}}^T = \begin{pmatrix}
    0     & -w'_z & w'_y  \\
    w'_z  & 0     & -w'_x \\
    -w'_y & w'_x  & 0
  \end{pmatrix}
\end{gather*}

And the a can then define the dual vectors too:
\begin{gather*}
  w = (w_x, w_y, w_z)\\
  w' = (w'_x, w'_y, w'_z)\\
\end{gather*}

We get the relationships:
\begin{gather*}
  \hat{w}q = w \cross q \\
  \hat{w}'q = w' \cross q \\
  \hat{w} = \mathcal{R}^T\hat{w}'\mathcal{R} \\
  \hat{w}' = \mathcal{R}\hat{w}
\end{gather*}

Defining $\textbf{V}$ as the speed of $S'$ as seen from $S$, we have the velocities (deriving the position one):
\begin{gather*}
  v=\mathcal{R}^T(v'+w'\cross r') + \textbf{V} \\
  v'=\mathcal{R}[v-\textbf{V}-w\cross(r-\textbf{R})]
\end{gather*}

Then with $\alpha$ as the angular acceleration, $A$ as the acceleration of $S'$ as seen by $S$, and deriving this once more we get the acceleration:
\begin{gather*}
  a' = -\alpha'\cross r' \text{; acimutal}\\
  \quad\quad - 2w'\cross v' \text{; coriolis}\\
  \quad\quad - w'\cross (w'\cross r') \text{; centrifugal}\\
  \quad\quad + \mathcal{R}a \\
  \quad\quad -\mathcal{R}\textbf{A} \text{; drag}\\
  \\
  a = \textbf{A} + \mathcal{R}^T[a' \\
  \quad\quad +\alpha'\cross r' \\
  \quad\quad + 1w'\cross v' \\
  \quad\quad + w'\cross (w'\cross r')] \\
\end{gather*}



\end{document}
