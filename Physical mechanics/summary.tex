\documentclass[a4paper,landscape,10pt]{cheatsheet}

\usepackage[spanish]{babel}
\usepackage[utf8]{inputenc}
\usepackage{physics}
\usepackage{amsmath}
\usepackage{bookmark}
\usepackage{amsfonts}
\usepackage{amssymb}
\usepackage{mathtools}
\usepackage{graphicx}
\usepackage{float}
\usepackage[rightcaption]{sidecap}

%addtolength{\oddsidemargin}{.875in} addtolength{\evensidemargin}{.875in} addtolength{\textwidth}{1.75in}
%addtolength{\topmargin}{.875in} \addtolength{\textheight}{1.75in}

\title{Physical mechanics}
\author{David Caro}
\date{09-01-2024}

\pdfinfo{%
  /Title    (Physical mechanics)
  /Author   (David Caro)
  /Creator  (David Caro)
  /Producer (David Caro)
  /Subject  (Physics)
  /Keywords ()
}

\begin{document}
\maketitle

%%%%%%%%%%%%%%%%%%%%%%%%%%%%%%%%%%%%%%%%%%%%%%%%%%%%%%%%%%%%%%%%%%%%%%%%%
\section{1 Cinematics}
Having:
\begin{itemize}
  \item $\mathcal{R}$ as rotation matrix of $S'$ seen from $S$
  \item $\bf{R}$ the relative position of $S'$ seen from $S$
\end{itemize}
the space coordinates of a point are:
\begin{gather*}
  r = \mathcal{R}^T r' + \bf{R} \\
  r'= \mathcal{R}(r - \bf{R})
\end{gather*}

We can then define the antisymmetric \textbf{angular momentum matrices} as:
\begin{gather*}
  \hat{w} = \mathcal{\dot{R}}^T\mathcal{R} = \begin{pmatrix}
    0    & -w_z & w_y  \\
    w_z  & 0    & -w_x \\
    -w_y & w_x  & 0
  \end{pmatrix} \\
  \hat{w}' = \mathcal{R}\mathcal{\dot{R}}^T = \begin{pmatrix}
    0     & -w'_z & w'_y  \\
    w'_z  & 0     & -w'_x \\
    -w'_y & w'_x  & 0
  \end{pmatrix}
\end{gather*}

And the a can then define the dual vectors too:
\begin{gather*}
  w = (w_x, w_y, w_z)\\
  w' = (w'_x, w'_y, w'_z)\\
\end{gather*}

We get the relationships:
\begin{gather*}
  \hat{w}q = w \cross q \\
  \hat{w}'q = w' \cross q \\
  \hat{w} = \mathcal{R}^T\hat{w}'\mathcal{R} \\
  \hat{w}' = \mathcal{R}\hat{w}
\end{gather*}

Defining $\textbf{V}$ as the speed of $S'$ as seen from $S$, we have the velocities (deriving the position one):
\begin{gather*}
  v=\mathcal{R}^T(v'+w'\cross r') + \textbf{V} \\
  v'=\mathcal{R}[v-\textbf{V}-w\cross(r-\textbf{R})]
\end{gather*}

Then with $\alpha$ as the angular acceleration, $A$ as the acceleration of $S'$ as seen by $S$, and deriving this once more we get the acceleration:
\begin{gather*}
  a' = -\alpha'\cross r' \text{; acimutal}\\
  \quad\quad - 2w'\cross v' \text{; coriolis}\\
  \quad\quad - w'\cross (w'\cross r') \text{; centrifugal}\\
  \quad\quad + \mathcal{R}a \\
  \quad\quad -\mathcal{R}\textbf{A} \text{; drag}\\
  \\
  a = \textbf{A} + \mathcal{R}^T[a' \\
  \quad\quad +\alpha'\cross r' \\
  \quad\quad + 1w'\cross v' \\
  \quad\quad + w'\cross (w'\cross r')] \\
\end{gather*}


\section{2 Dynamics}
\subsection*{2.1 Newton dynamics}
The three newton laws:
\begin{itemize}
  \item In absence of forces, a body remains it's constant lineal movement $\bf{p} = m\bf{v}$:
        $$
          \bf{0} = \derivative{}{t}\bf{p} \longrightarrow \bf{p} = \text{ct.}
        $$
  \item The change of lineal momentum is given by the force that acts on a body:
        $$
          \bf{F} = \derivative{}{t}\bf{p}; \quad m=\text{ct.} \rightarrow \bf{F} = m\bf{a}
        $$
  \item Given two particles that interact, the force on particle 1 is the same strength and opposite sign than the force
        on particle 2.
        $$
          \bf{F}_1 = -\bf{F}_2
        $$
\end{itemize}

\subsection*{2.2 Common differential equations and solutions}
Common types of problems and their differential equations and solutions:
\begin{itemize}
  \item Uniform rectilinear movement:
        $$
          \ddot{q} = 0 \longrightarrow q(t) = q_0 + \dot{q}_0 t
        $$
  \item Uniformly accelerated movement:
        $$
          \ddot{q} = a \longrightarrow q(t) = q_0 + \dot{q}_0 t + \frac{1}{2}at^2
        $$
  \item Uniformly accelerated with friction:
        $$
          \ddot{q} = a - b\dot{q} \longrightarrow \dot{q}(t) = \frac{a}{b} + \left(\dot{q}_0 - \frac{a}{b}\right)e^{-bt}
        $$
        Where $a/b$ is the limit speed.
  \item Harmonic oscillator:
        \begin{gather*}
          \ddot{q} + \omega^2 q=0 \rightarrow                             \\
          q(t)=\frac{\dot{q}_0}{\omega}\sin(\omega t) + q_0\cos(\omega t) \\
          \quad = A\sin(\omega t+\phi)
        \end{gather*}
        Where
        \begin{gather*}
          A=\sqrt{q^2_0 + \frac{\dot{q}^2_0}{\omega^2}} \\
          \phi = \arccos\frac{\dot{q}_0}{\sqrt{q^2_0\omega^2+\dot{q}^2_0}}
        \end{gather*}
  \item Charged particle in an electric and magnetic field:
        \begin{gather*}
          \bf{F} = q\bf{E}+q\bf{v}\cross\bf{B} \quad\text{(Lorentz force)} \\
        \end{gather*}
        Ends up creating a spiral with the axes perpendicular to the magnetic field, that with a frequency $\omega_c$
        called \textbf{cyclotronic frequency}, and a constant modulo of the speed.
\end{itemize}

\subsection*{2.3 Non-inertial reference frames}
When applying newton dynamics from the point of view of a non-inertial reference frame, we get fictional forces:
\begin{gather*}
  m\bf{a}' = \bf{F}' \quad ;\mathcal{R}\bf{F}\\
  \qquad -m\mathcal{R}\bf{A} \quad ;\bf{F}'_\text{drag}\\
  \qquad -m\boldsymbol{\dot{\omega}}'\cross \bf{r}' \quad ;\bf{F}'_\text{acimutal}\\
  \qquad - 2m\boldsymbol{\omega}'\cross\bf{v}' \quad ;\bf{F}'_{\text{Coriolis}}\\
  \qquad -m\boldsymbol{\omega}'\cross(\boldsymbol{\omega}'\cross\bf{r}')  \quad ;\bf{F}'_{\text{centrifugal}}
\end{gather*}

Then, for a particle on the surface of earth we have:
\begin{gather*}
  \bf{a}' = \frac{1}{m}\mathcal{R}\bf{F}_\text{ng} + \textbf{g}'_\text{effective} - 2\boldsymbol{\omega}'\cross \textbf{v}' \\
\end{gather*}
where $\bf{F}_\text{ng}$ is the non-gravitational force and the effective gravitational acceleration is:
\begin{gather*}
  \bf{g}'_\text{effective} = (-\omega^2_T R_T \cos\lambda\sen\lambda)\textbf{e}'_y \\
  \quad + (-g + \omega^2_T R_T \cos^2\lambda)\bf{e}'_z
\end{gather*}
Where $\lambda$ is the latitude.
\end{document}
