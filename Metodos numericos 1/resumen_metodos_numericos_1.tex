\documentclass[a4paper,landscape,10pt]{cheatsheet}

\usepackage[spanish]{babel}
\usepackage[utf8]{inputenc}
\usepackage{physics}
\usepackage{amsmath}
\usepackage{bookmark}
\usepackage{amsfonts}
\usepackage{amssymb}
\usepackage{mathtools}
\usepackage{graphicx}
\usepackage{float}

%addtolength{\oddsidemargin}{.875in}
%addtolength{\evensidemargin}{.875in}
%addtolength{\textwidth}{1.75in}
%addtolength{\topmargin}{.875in}
%\addtolength{\textheight}{1.75in}

\title{Métodos Numéricos I}
\author{David Caro}
\date{04-23-2022}

\pdfinfo{%
  /Title    (Métodos Numéricos I)
  /Author   (David Caro)
  /Creator  (David Caro)
  /Producer (David Caro)
  /Subject  (Física)
  /Keywords ()
}

\begin{document}
\maketitle

%%%%%%%%%%%%%%%%%%%%%%%%%%%%%%%%%%%%%%%%%%%%%%%%%%%%%%%%%%%%%%%%%%%%%%%%%
\section{Ecuaciones diferenciales de primer orden}
\textbf{Variables separadas}:\\
$\qquad f(y)dy=g(x)dx$\\
Solución:\\
$\qquad \int f(y)dy=\int g(x)dx$\\
$\qquad F(y)=H(x)+C$\\

\hfill\\
\textbf{Reducibles a variables separadas}:\\
$\qquad \frac{dy}{dx}=f(ax+by+c)$\\
Solución, cambio de variable:\\
$\qquad z=ax+by+c$\\
$\qquad y=\frac{z-ax-c}{b}$\\
$\qquad \frac{dy}{dx}=\frac{1}{b}\left(\frac{dz}{dx}-a\right)$\\
$\qquad \frac{1}{b}\left(\frac{dz}{dx}-a\right)=f(z)$  <- Variables separadas!\\
\qquad\textbf{Nota:} deshacer el cambio de variable al final ($z=ax+by+c$)


\hfill\\
\textbf{Exactas}:\\
$\qquad M(x,y)dx+N(x,y)dy=0$\\
$\qquad :\frac{\partial M(x,y)}{\partial y}=\frac{\partial N(x,y)}{\partial x}$\\
Solución:\\
$\qquad \int M(x,y)dx + \int\left(N(x,y)-\frac{\partial}{\partial y}\int M(x,y)dx\right)dy = 0$\\

\hfill\\
\textbf{Reducibles a exactas}:\\
$\qquad M(x,y)dx+N(x,y)dy=0$\\
$\qquad :\frac{\partial M(x,y)}{\partial y}\neq\frac{\partial N(x,y)}{\partial x}$\\
Solución:\\
$\qquad \mu(x,y)M(x,y)dx+\mu(x,y)N(x,y)dy=0$\\
\begin{itemize}
  \item Si $\frac{1}{N}\left[\frac{\partial M}{\partial y}-\frac{\partial N}{\partial x}\right]=f(x)$\\
        $=> \mu(x,y)=e^{\int f(x)dx}$
  \item Si $-\frac{1}{M}\left[\frac{\partial M}{\partial y}-\frac{\partial N}{\partial x}\right]=f(y)$\\
        $=> \mu(x,y)=e^{\int f(y)dy}$
\end{itemize}

\hfill\\
\textbf{Lineales}:\\
$\qquad a_1(x)\frac{dy}{dx}+a_0(x)y=b(x)$\\
Solución:\\
\begin{itemize}
  \item $a_0=0 => y(x)=\int\frac{b(x)}{a_1(x)}dx$ (siempre que $a_1(x)$ no se anule)\\
  \item $a_0=a'_1 $\\$=> y(x)=\frac{1}{a_1(x)}\{\int b(x)dx+C\}$\\
\end{itemize}

\hfill\\
\textbf{Reducibles a lineales}:\\
$\qquad a_1(x)\frac{dy}{dx}+a_0(x)y=b(x) \equiv \frac{dy}{dx}+P(x)y=Q(x)$\\
Solución (resolviendo como exacta, pero queda una fórmula más sencilla):\\
$\qquad y(x)=$\\
$\qquad e^{-\int P(x)dx}\cdot\left[\int e^{\int P(x)dx}\cdot Q(x)dx + C\right]$\\

\hfill\\
\textbf{De Bernoulli}:\\
$\qquad \frac{dy}{dx}+P(x)y=Q(x)y^n$\\
Solución:\\
\begin{itemize}
  \item Si $n\in\{0,1\}$ -> lineal\\
  \item Si no, sustituir $v=y^{1-n} \implies \frac{dv}{dx}=(1-n)y^-n\frac{dy}{dx}$
        \\ Quedando la ecuacion lineal:
        \\$\frac{1}{1-n}\frac{dv}{dx}+P(x)v=Q(x)$
\end{itemize}

\hfill\\
\textbf{De Riccati}:\\
$\qquad \frac{dy}{dx}+P(x)y+Q(x)y^2=f(x)$\\
Solución:\\
\qquad\textbf{Necesitamos} saber una solución $y_1(x)$\\
\qquad Usamos el cambio de variable $y=y_1+z$\\
\qquad Queda una ecuación de Bernoulli: \\
$\qquad\quad \frac{dz}{dx}+\left[P(x)+2y_1(x)\right]z=-Q(x)z^2$\\

\hfill\\
\textbf{Homogéneas}:\\
$\qquad \frac{dy}{dx}=f(x,y)$\\
$\qquad\quad | \quad f(x,y)=f(\lambda x, \lambda y)$\\
Solución:\\
$\qquad z=\frac{y}{x}$ -> Se convierte en variables separadas\\
$\qquad\qquad \frac{dy}{dx}=v+x\frac{dv}{dx}$

\hfill\\
\textbf{Reducibles a homogéneas}:\\
$\qquad \frac{dy}{dx}=f(\frac{a_1x+b_1y+c_1}{a_2x+b_2y+c_2})$\\
Solución:\\
\qquad Usar el cambio de variables:\\
$\qquad\qquad X=x-x_0$\\
$\qquad\qquad Y=y-y_0$\\
\qquad Donde $(x_0,y_0)$ es el punto de intersección de:\\
$\qquad\begin{cases}
    a_1x+b_1y+c_1=0 \\
    a_2x+b_2y+c_2=0 \\
  \end{cases}
$\\


\hfill\\
\textbf{De Clairaut}:\\
$\qquad y=x\frac{dy}{dx}+f(\frac{dy}{dx})$\\
Solución:\\
\begin{itemize}
  \item Al derivar respecto de x:\\
        $x+f'(\frac{dy}{dx})\frac{d^2y}{dx^2}=0$\\
  \item queda que:\\
        \begin{itemize}
          \item $\frac{d^2y}{dx}=0\implies\frac{dy}{dx}=c$
          \item ó $f'(\frac{dy}{dx})=-x$
        \end{itemize}
  \item Sustituyendo la primera tenemos las rectas envolventes (\textbf{solución general}):\\
        $y=cx+f'(c)$\\
  \item Sustituyendo la segunda tenemos la \textbf{solución singular} paramétrica:\\
        $
          \begin{cases}
            x(p)=-f'(p)      \\
            y(p)=f(p)-pf'(p) \\
          \end{cases}
        $\\
        Despejando $x$ respecto $p$ y sustituyendo obtendremos una fórmula dependiente de $x$ e $y$.
\end{itemize}


\hfill\\
\textbf{Trayectorias ortogonales}:\\
Dado $F(x,y)=C$, las trayectorias ortogonales vienen dadas por la solución de la EDO:\\
$\qquad \frac{dy}{dx}=\frac{F_x(x,y)}{F_y(x,y)}$\\


\end{document}